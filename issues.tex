% Plantilla simple para tareas de la Licenciatura en Física
% Fer Flores - Universidad de Guadalajara - Noviembre 2023

%%%%%%%%%%%%%%%%%%%%%%%%%%%%%%%%%%%%%%%%%%-PREÁMBULO-%%%%%%%%%%%%%%%%%%%%%%%%%%%%%%%%%%%%%%%%%%

% Paqueterías

\documentclass{assignment}
\usepackage[pdftex]{graphicx} % FIGURAS
\usepackage{xcolor}
\definecolor{LightGray}{gray}{0.95}
\usepackage{fancyvrb, minted} % CÓDIGO
\usepackage[letterpaper, margin = 2.5cm]{geometry} % TAMAÑO DE PÁGINA Y MÁRGENES
\usepackage[T1]{fontenc} % Importante para acentos automáticos y símbolos de escritura
\usepackage[spanish, mexico]{babel} % Importante para Español
\usepackage{amsmath, amsfonts, amssymb} % Ecuaciones, caracteres y símbolos especiales
\usepackage{hyperref, url}  % Links y Hyperlinks en el documento
\usepackage{fancyhdr}
\usepackage{ctex}

%-----------------------------------------ETIQUETAS--------------------------------------------

\student{张轩华}                             % NOMBRE
\semester{2023第一学期}                                % SEMESTRE (202X A/B)
\date{2023 年 12 月 11 日}                                   % Fecha (Modifica a DD/MM/AAAA)

\courselabel{心理}          % CLAVE Y MATERIA
\exercisesheet{自我调节}{总结和方案}     % NÚMERO Y TÍTULO DE LA TAREA(任务编号和标题)

\school{高一}          % CARERA (Física, la mejor carrera)(职业(物理,最佳职业))
\university{宁海海亮}         % LA PODEROSÍSIMA

%%%%%%%%%%%%%%%%%%%%%%%%%%%%%%%%%%%%%%%%%%-DOCUMENTO-%%%%%%%%%%%%%%%%%%%%%%%%%%%%%%%%%%%%%%%%%%%%

\begin{document}

%-----------------------------------------First Page-------------------------------------------
\begin{problem}

\section{Abstract}

\noindent 关于问题的概要

\begin{enumerate}
    \item 缺乏对他人情感的理解力
    \item 语言和交流
    \item 对细节的\emph{机械记忆}\footnote[1]{不加理解只是简单重复的记忆}
    \item 行为公式化
    \item 长期以往的焦虑
    \item “\emph{半解离}\footnote[2]{解离症意指记忆、自我意识或
    认知的功能上的崩解,这里指短时间的解离状态}”
    \item 不适当的单方面的社会交往,缺少建立友谊的能力从而导致
    \emph{社会隔离}\footnote[3]{个体在人际关系上处于
    孤立或被隔离状态,例如试图以"武力"争得别人的注意,当然会变得愈加孤立}
  \end{enumerate}

\noindent 曾被确诊的疾病:
  \begin{enumerate}
    \item 阿斯伯格综合征(自闭症谱系)
    \item 注意力缺陷及多动障碍(ADHD)
    \item 抑郁症
    \item 焦虑症
  \end{enumerate}
%------------------------------------------------------------------------------------------
\pagebreak

\subsection{问题的分述}

\noindent 关于被列举的问题的一些细节
    \begin{itemize}
    \item Q1:缺乏对他人情感的理解力\\
    情感是人对客观事物是否满足自己的需要而产生的态度体验\\
    换句话说,你不去亲身体验自然不知道事物究竟如何,应该多去体验生活。
    \item Q2:语言和交流
    \begin{enumerate}
        \item 语言缺乏连贯性和交互性\\%演讲一样
                \emph{连贯性},语言经常是离题和带偶然性的给人以一种松散和
                缺乏内在联系和连贯性的感觉\\
                \emph{交互性},例如喋喋不休地向“他人”进行演说,
                但话题范围狭窄,这自然会引起他人反感。


                总而言之,就是一种以自我为中心的交谈模式所造成的,
                不能清除界定话题的变化,
                不能制止说出内心的想法以及兴趣的狭隘。

        \item 非语言交流贫乏
    \end{enumerate}
    \item Q3:对细节的机械记忆\\
            例如背地图,过分在意一些细节等\\
            不见得在情绪上有什么消极影响,只是加重了精神疲劳。
    \item Q4:行为公式化\\
            行为反应强烈依赖公式化和刻板的社会行为规范和社会规则
            而不能以直觉和自发的形式理解别人的意图。\\
            其实我自幼如此,甚至是影响到人格特质。\\
            虽然成因复杂,但不见得有严重的影响,只是在别人看来
            略显古怪罢了。
    \item Q5:长期以往的焦虑\\
            急性焦虑症\\
            主要表现为\emph{各种运动性不安}和\emph{呼吸性碱中毒}\\
            由于呈现出病理性的特点,可能是目前最为严重的问题,
            目前打算通过定期的放松训练和药物来改善。

    \item Q6: “半解离”\\
            这是我长期以往的一个习惯:当处于高压力和不稳定的情绪中时,
            通过减弱和改变自我认知来屏蔽外界刺激和情绪负担,
            目的在于保护自己的人格,或者理解为自我封闭。


            由于改变了部分认知,造成了一种“失自我感”,甚至
            是“自我认同混乱”。从我自己的角度看,类似于
            一种“人格异常\footnote[1]{在“认知、情绪反应、人际关系与冲
            动控制”四个方面中,起码有两个以上出现长时间持续存在的执拗
            与功能损害,且与其文化背景所预期的,偏离甚远}”,
            于外人来看,可能类似于多重人格或者“显得很装”。


            实验表明\emph{“情感事件构成了我们最强烈、最有价值的记忆”}
            \footnote[2]{出自期刊《Nature Human Behaviour》,详见\cite{1}},
            所以,出于认知的变化,我难以记得那些使我产生情绪的事情,
            但这些“情感”通过一个类似于人格的东西被保留了下来,
            造成了“情感”的堆积。
    \item Q7:社会隔离问题\\
            “解离”当然是一个很大的因素(详见Q6)\\
            但造成这个问题的根本,在于同
            社会期望\footnote[1]{群体根据个体的社会角色及身份,对其提出的希望和要求}
            的差异\\
            比如学生应当交作业、积极听课,若不曾遵守不免让人嗤之以鼻


            然而这个问题在我身上并不只是体现在学校,目前的办法
            就是去学习比人如何做,以免自己行为出格。
    \end{itemize}

\section{放松策略}
\noindent 这里将分成对外界反馈的等级和具体的活动两部分\footnote[2]{这里的“压力情境”和“情绪强度”出自于\href{https://zhuanlan.zhihu.com/p/466825383?utm_psn=1717710121701052416}{情绪管理策略}}
\subsection{压力情境}
\begin{itemize}
    \item 等级一\\
    我感觉如何:太棒了!我在做我喜欢的事。非常放松。
    我应该怎么做:享受这一刻并把这时的感觉记下来,列出是什么让我有这样的感觉。
    \item 等级二\\
    我感觉如何:不好也不坏。
    我应该怎么做:保持这种状态。记住每天都要做放松练习。
    \item 等级三\\(能清晰且理智地思考的最后机会)
    我感觉如何:紧张。觉得生活不公平。总想一些负面的事。
    我应该怎么做:这是危险地带。此时最好离开,回家或散步都好。深呼吸。不要与别人交流,不要给别人发信息。
    \item 等级四\\
    我感觉如何:生气或非常难受。
    我应该怎么做:离开此地!走到安静的地方,比如卧室、浴室或学习的地方。
    \item 等级五\\
    我感觉如何:我失去控制了!
    我应该怎么做:立刻从我确认为安全与信任的人那里寻求帮助。    
\end{itemize}

\subsection{情绪强度}
\begin{itemize}
    \item 等级一\\
    我的感受:我感觉不错且放松。了解行程与别人对我的期望。我做好了上课或上班的准备。
    我能做些什么:早上离开家之前,事先决定好用哪种活动来放松,这让我感觉很好。有时,在午餐时间冥想也能让我放松下来。
    \item 等级二\\
    我的感受:我感觉有点紧张,就像有一天别人要我去做一件很难的事情。
    我能做些什么:在离开家之前,做一些放松活动。我明白今天可能会不轻松,我会在手机备忘录里写下一些自我肯定的话,一天内读上几遍。
    \item 等级三\\(能清晰且理智地思考的最后机会)
    我的感受:我肚子痛、肌肉紧张、头痛,感到紧张与轻微的害怕。
    我能做些什么:这是使用放松技巧的的最佳时机。如果可能的话,离开这个让你有压力的情境。然后深呼吸,进行冥想。想象一下自己喜欢的地方。一直放松到冷静下来为止。
    \item 等级四\\
    我的感受:感觉自己快要被负面情绪淹没了,我真想与人大吵一架。
    我能做些什么:暂停与别人的交流(比如打电话、讨论、玩电脑游戏等)。保持安静、不说话,离开这里去一个隐蔽些的地方。
    \item  等级五\\
    我的感受:好恐怖!我觉得自己失去控制了。我想破坏东西。
    我能做些什么:保持安静、不说话。闭上眼睛,试着放慢呼吸。不去看让我生气的人或事。
\end{itemize}





\begin{minted}[frame=lines, linenos, bgcolor=LightGray]{python}
    
    一些能够让我放松的活动:
    听音乐
    描摹有创意的icon图标


\end{minted}

\end{problem}
%---------------------------------------------------------------------------------------------
% \begin{problem}

% \section{Segundo Problema}

% \noindent
    
% \end{problem}
%---------------------------------------------------------------------------------------------



%--------------------------------------BIBLIOGRAFIA-------------------------------------------

\newpage

\nocite{*} % Agrega las referencias aunque no las hayas citado directamente

%参考文献
\bibliographystyle{unsrt}    % ESTILO DE BIBLIOGRAFÍA (Recomendados: abbrv, ieeetr, apalike, unsrt)
\bibliography{refs}     % REFERENCIAS EN ARCHIVO SEPARADO


\end{document}
